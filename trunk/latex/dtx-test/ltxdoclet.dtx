% \iffalse meta-comment
%
%%  (C) 2008 Paul Ebermann
%%
%%   Package ltxdoclet - Dokumentation von Java-Paketen.
%%
%%   Die Datei ltxdoclet.dtx sowie die dazugehörige
%%   ltxdoclet.ins sowie die damit generierte
%%   ltxdoclet.sty stehen unter der
%%   "LaTeX Project Public License" (LPPL, zu finden
%%   unter http://www.latex-project.org/lppl/, sowie
%%   auch in den meisten TeX-Distributionen in
%%   texmf/docs/latex/base/lppl*.txt), Version 1.3b oder
%%   später (nach Wahl des Verwenders).
%%
%%   Der 'maintenance-status' ist (zur Zeit) 'author-maintained'.
%%   
%%   Das heißt u.a., die Dateien dürfen frei vertrieben werden,
%%   bei Änderungen (durch andere Personen als Paul Ebermann)
%%   ist aber der Name der Datei zu ändern.
%
% \fi
%
% \iffalse
% ---------------------------------------------
%<package>\NeedsTeXFormat{LaTeX2e}[2003/12/01]%
% -----------------------------------________________------------------
%<package>\ProvidesPackage{ltxdoclet}[2010/02/14 v0.0 Dokumentation von Java-Paketen (PE)]
% ------------------------------------________________-----------------
% \fi
%
% \iffalse
%<*driver>
\documentclass{scrartcl}
%
% Esperantajn tekstojn oni prefere entajpu per unikodaj literoj.
\RequirePackage[utf8x]{inputenc}
\RequirePackage[ngerman]{babel}
\addto{\extrasesperanto}{%
  \renewcommand*{\generalname}{\^Generale}%
  \GlossaryPrologue{%
    \section{\^San\^goj}%
  }%
  \GlossaryMin=3\baselineskip
  \IndexPrologue{%
    \section{Indekso}
    Kurzivaj nombroj indikas la lokojn, kie la indeks-ero estas priskribita,
    substrekitaj nombroj indikas la lokon de la difino, la aliaj estas uzoj.
  }
  \IndexMin=5\baselineskip
}

\usepackage{ltxdoclet}

\usepackage[bookmarks=false]{hyperref}
\usepackage[countalllines, withmarginpar]{gmdoc}
\let\pack\textrm%
\renewcommand*{\EOFMark}{}

\RecordChanges
\begin{document}
   % ^ ne taŭgas kiel specialsigno, ĉar ni
   % bezonas ĝin por ^^B, ^^M ktp.
   \catcode`\^=7%
   \MakeShortVerb\'
   \DocInput{ltxdoclet.dtx}
\end{document}
%</driver>
% \fi
%
% \CheckSum{0}
%
% \GetFileInfo{ltxdoclet.sty}
%
%
% \title{Das \pack{ltxdoclet}-Paket: Java-Dokumentation (und Quelltexte) in \LaTeX. \
%   \thanks{ Dieses Dokument gehört zu \
%    \pack{ltxdoclet}~\fileversion, von~\filedate.}}
% \author{Paul Ebermann\thanks{\texttt{Paul-Ebermann@gmx.de}}}
%
% \maketitle
% 
%
% \tableofcontents
%
% \section{Uzanta dokumentaĵo}
%
%
% \StopEventually{\PrintChanges\PrintIndex}
%
% \section{Implementado}
%
%<*package>
%
%
% \subsection{Opcioj}
%
% Jam fino :-)
\endinput
%</package>
% (Jes, vere fino.)
%
% \Finale
%\endinput


%%% Folgendes ist nur für meinen Editor.
%%%
%%% Local Variables:
%%% mode: docTeX
%%% TeX-master: t
%%% End:
