% \iffalse meta-comment
%
%%  (C) 2008 Paul Ebermann
%%
%%   Package ltxdoclet - Dokumentation von Java-Paketen.
%%
%%   Die Datei ltxdoclet.dtx sowie die dazugehörige
%%   ltxdoclet.ins sowie die damit generierte
%%   ltxdoclet.sty stehen unter der
%%   "LaTeX Project Public License" (LPPL, zu finden
%%   unter http://www.latex-project.org/lppl/, sowie
%%   auch in den meisten TeX-Distributionen in
%%   texmf/docs/latex/base/lppl*.txt), Version 1.3b oder
%%   später (nach Wahl des Verwenders).
%%
%%   Der 'maintenance-status' ist (zur Zeit) 'author-maintained'.
%%   
%%   Das heißt u.a., die Dateien dürfen frei vertrieben werden,
%%   bei Änderungen (durch andere Personen als Paul Ebermann)
%%   ist aber der Name der Datei zu ändern.
%
% \fi
%
% \iffalse
% ---------------------------------------------
%<package>\NeedsTeXFormat{LaTeX2e}[2003/12/01]%
% -----------------------------------________________------------------
%<package>\ProvidesPackage{ltxdoclet}[2010/02/14 v0.0 Dokumentation von Java-Paketen (PE)]
% ------------------------------------________________-----------------
% \fi
%
% \iffalse
%<*driver>
\documentclass[dvipsnames]{scrartcl}
%
% Esperantajn tekstojn oni prefere entajpu per unikodaj literoj.
\RequirePackage[utf8x]{inputenc}
\PrerenderUnicode{}
\RequirePackage[ngerman]{babel}
\addto{\extrasesperanto}{%
  \renewcommand*{\generalname}{\^Generale}%
  \GlossaryPrologue{%
    \section{\^San\^goj}%
  }%
  \GlossaryMin=3\baselineskip
  \IndexPrologue{%
    \section{Indekso}
    Kurzivaj nombroj indikas la lokojn, kie la indeks-ero estas priskribita,
    substrekitaj nombroj indikas la lokon de la difino, la aliaj estas uzoj.
  }
  \IndexMin=5\baselineskip
}

\usepackage[dvipsnames]{xcolor}
\usepackage{ltxdoclet}
%! \usepackage[scaled]{luximono}
\usepackage[bookmarks=false]{hyperref}
\usepackage[countalllines, withmarginpar]{gmdoc}
\usepackage[inline, visible]{gmdoc-enhance}
\let\pack\textrm%
\renewcommand*{\EOFMark}{}

\RecordChanges
\begin{document}
   % ^ ne taŭgas kiel specialsigno, ĉar ni
   % bezonas ĝin por ^^B, ^^M ktp.
   \catcode`\^=7%
   \MakeShortVerb\'
   \DocInput{ltxdoclet.dtx}
\end{document}
%</driver>
% \fi
%
% \CheckSum{0}
%
% \GetFileInfo{ltxdoclet.sty}
%
%
% \title{Das \pack{ltxdoclet}-Paket: Java-Dokumentation (und Quelltexte) in \LaTeX. \
%   \thanks{ Dieses Dokument gehört zu \
%    \pack{ltxdoclet}~\fileversion, von~\filedate.}}
% \author{Paul Ebermann\thanks{\texttt{Paul-Ebermann@gmx.de}}}
%
% \maketitle
% 
%
% \tableofcontents
%
% \section{Nutzerdoku}
%
%
% \StopEventually{\PrintChanges\PrintIndex}
%
% \section{Implementation}
%
%<*package>
%
%
% \subsection{Package-Optionen}
%
%  Bisher gibt es keine.
%
% \subsection{Geladene Pakete}
%
% Wir laden das \pack{color}-Paket, um Farben verwenden zu können.
%!\RequirePackage[dvipsnames]{color}

%
% \subsection{Diverse Makros}
%
% \begin{macro}{\noprint}
%   '\noprint' ist ein Klon des '\@gobble'-Makros aus dem \LaTeX-Kernel.
%   Es dient dazu, im generierten \LaTeX-Quelltext Debug-Informationen auszugeben,
%   ohne dass sie in der Ausgabe auftauchen.
\newcommand*\noprint[1]{}%
% \end{macro}
%
% \subsection{Text ausrichten}
%
% \begin{macro}{\clap}
%   Dieses Makro habe ich aus \pack{mathtools} geklaut. Es ist ein Verwandter 
%   der bekannten '\llap' und '\rlap'. Es setzt eine horizontale Box mit dem
%   Argument als Inhalt zentriert an der aktuellen Stelle, ohne dass sie Platz
%   verwendet.
\newcommand*\clap[1]{%
  \hb@xt@\z@{\hss#1\hss}
}
% \end{macro}
%
% \begin{macro}{\clapon}\marg{text1}\marg{text2}
%   Verwandt mit '\clap', zentriert dieses Makro \meta{text1} nicht
%   mit Breite $0$, sondern über \meta{text2}. (D.h. beides wird gesetzt,
%   relativ zueinander zentriert, und das Ergebnis hat die Breite von
%   \meta{text2}.) 
\providecommand*{\clapon}[2]{%
  \setbox\@tempboxa\hbox{#2}% Wir merken uns \meta{text2} in einer Box.
  \hbox to\wd\@tempboxa{% Dann öffnen wir eine Box mit der
% Breite von \meta{text2}, \dots
    \hss#1\hss}% \dots und setzen darin \meta{text1}, mit beidseitig
% flexiblem Platz (d.h. zentriert).
  \kern-\wd\@tempboxa% Dann gehen wir wieder zurück
% zum Anfang (mit einem negativen Abstand).
  \unhbox\@tempboxa}% und hier setzen wir \meta{text1},
% außerhalb seiner Box. (Es ist zu überlegen, ob statt '\unhbox'
% eher '\box' sinnvoller wäre, denn nun kann \meta{text2} noch
% vom Zeilenpasser bearbeitet werden, und damit die Zentrierung
% kaputtgehen.)
% \end{macro}
%
% \subsection{Literale hervorheben}
%

% \begin{macro}{\markString}
% \begin{macro}{\markNumber}
% \begin{macro}{\markLiteralKeyword}
%   Diese drei Makros sind Deklarationen, die
%   den Bereich bis zum nächsten Gruppenende als
%   Literal markieren. Sie werden von unserem
%   Quelltext-Drucker in der Form '{\markNumber 20}' verwendet: {\markNumber '20'}
\providecommand*{\markString}{%
  \color{blue}}% Strings markieren wir blau.
\providecommand*{\markNumber}{%
  \color[named]{ForestGreen}}% Zahlen sind grün.
\providecommand*{\markLiteralKeyword}{%
  \color[named]{Brown}}% Und Schlüsselwortliterale wie {\markLiteralKeyword 'null'},
% {\markLiteralKeyword 'true'}, {\markLiteralKeyword 'false'} sind braun.
% \end{macro}
% \end{macro}
% \end{macro}
%
%
% \subsection{sourcecode-Umgebung}
%
% \begin{environment}{sourcecode}
% Diese Umgebung verwenden wir, um Quelltext zu setzen.

%\begin{minipage}{.45\textwidth}
%\begin{sourcecode}
%\textbf{private} DocletStart()
%\{
%~~~~super()~\clap{\textbf{;}} 
%\}
%\end{sourcecode}
%\end{minipage}
%\begin{minipage}{.45\textwidth}
%\begin{verbatim}
%\begin{sourcecode}
%\textbf{private} DocletStart()
%\{
%~~~~super()~\clap{\textbf{;}} 
%\}
%\end{sourcecode}
%\end{verbatim}
%\end{minipage}
%

% Der Inhalt dieser Umgebung wird von unserem \LaTeX-doclet
% automatisch generiert, indem ein Syntaxbaum des Compilers
% abgelaufen wird.
%
% Hier die Definition:
\newenvironment*{sourcecode}%
{% Einstellungen am Anfang:
  \par% zuerst beenden wir den Absatz, falls da einer ist.
  \setlength{\parindent}{0pt}%  Wir wollen keine Paragraphen-Einrückung hier.
  \ttfamily\small% Alles soll in einer \texttt{nichtproportionalen} Schrift und
  %  etwas kleiner sein.
  %
  \obeyspaces\obeylines% Leerzeichen und Zeilenumbrüche
  % sollen bitte behalten werden. (Alternativ könnten wir
  % unseren Quelltext-Formatierer immer '\par' ausgeben lassen,
  % aber so ist es einfacher.)
  %
  \raggedright% das bringt etwas weniger \emph{Overfull hbox}-Meldungen, und
  % erlaubt einen gewissen Zeilenumbruch im Quelltext. Wir müssen uns da aber
  % noch etwas besseres ausdenken.
  %
}{% Am Ende der Umgebung muss nicht so viel gemacht werden:
  \par% wir beenden nur noch den Absatz.
}%
%
% \end{environment}

%
% \subsection{Ende}
% 
%

\endinput
%</package>
% (Jes, vere fino.)
%
% \Finale
%\endinput


%%% Folgendes ist nur für meinen Editor.
%%%
%%% Local Variables:
%%% mode: docTeX
%%% TeX-master: t
%%% End:
